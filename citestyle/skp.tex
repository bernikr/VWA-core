% skpcite script for use at Sir-Karl-Popper-Schule by Valentin Huebner

\usepackage[backend=biber, style=authoryear, refsection=none, maxcitenames=1, maxbibnames=99]{biblatex}
% Im Fliesstext kein Trennzeichen zwischen Name und Jahr
\DeclareFieldFormat{postnote}{#1}
% Im Fliesstext Doppelpunkt vor Jahr
\renewcommand{\postnotedelim}{: }
% Titel ohne Anführungszeichen und nicht kursiv
\DeclareFieldFormat{title}{#1}
%  Alles ohne Anführungszeichen und kursiv
\DeclareFieldFormat{title}{\mkbibemph{#1}}
%  Artikel ohne Anführungszeichen und nicht kursiv
\DeclareFieldFormat[article]{title}{#1}
%  Dissertation ohne Anführungszeichen und nicht kursiv
\DeclareFieldFormat[thesis]{title}{#1}
%  Dissertation ohne Anführungszeichen und nicht kursiv
\DeclareFieldFormat[incollection]{title}{#1}
% Autor in Kapitaelchen
\DeclareFieldFormat[book]{edition}{(#1)} % Ausgabe bei Büchern in Klammer
\renewcommand{\mkbibnamefirst}[1]{\textsc{#1}}
% s.o.
\renewcommand{\mkbibnamelast}[1]{\textsc{#1}}
% s.o.
\renewcommand{\mkbibnameprefix}[1]{\textsc{#1}}
% s.o.
\renewcommand{\mkbibnameaffix}[1]{\textsc{#1}}

% "In:" bei Journal-Artikeln entfernt,
\renewbibmacro{in:}{
	% aber für Sammelbaende beibehalten
	\ifentrytype{article}{}{\printtext{\bibstring{in}\intitlepunct}}}
	% Journaltitel kursiv und gefolgt von Beistrich
\DeclareFieldFormat{journaltitle}{\mkbibemph{#1},}

\renewcommand{\mkbibbrackets}[1]{#1}

\makeatletter
\renewcommand\listoffigures{%
	\chapter{\listfigurename}% Used to be \section*{\listfigurename}
	\@mkboth{\MakeUppercase\listfigurename}%
	        {\MakeUppercase\listfigurename}%
	\@starttoc{lof}%
}
\makeatother

\addto\captionsgerman{\renewcommand{\figurename}{Abb.}}

\DefineBibliographyStrings{ngerman}{andothers={et~al\adddot}}

\DeclareCiteCommand{\tabcite}
	{\usebibmacro{prenote}}
	{\usebibmacro{citeindex}}
	{}
	{\usebibmacro{postnote}}

%justbecausevalentin wantedtodoit:
\newcommand{\skpciteresource}[1]{
  \addbibresource{#1}
}

%cite  (\skpcite{Name der Quelle im .bib file}{Seitenzahl})
\newcommand{\skpcite}[2]{\qvlink{\parencite[vgl.][#2]{#1}}}

\newcommand{\Skpcite}[2]{\qvlink{\parencite[Vgl.][#2]{#1}}}

%multicite \skpcitemulti[42]{Mixer202}[18]{Baron1992} beliebig viele möglich!
\newcommand{\skpcitemulti}{\parencites(vgl.)()}

\newcommand{\Skpcitemulti}{\parencites(Vgl.)()}


%ref
\newcommand{\skpref}[1]{\qvlink{(\cite[vgl.][]{#1})}}

\newcommand{\Skpref}[1]{\qvlink{(\cite[Vgl.][]{#1})}}

%Literaturverzeichnis (in .bib file: Keywords={lit} für Bücher und co und Keywords={web} für Internetquellen)
\newcommand{\skpciteprintlit}{
	\defbibheading{lit}{\chapter{Quellenverzeichnis}}
	\printbibliography[heading=lit, keyword=lit]
	\label{quellenverzeichnis}
}

\makeatletter
\newcommand*{\tofnumberline}[1]{\hb@xt@7em{Abbildung~#1:\hfil}}
\makeatother

\makeatletter
\renewcommand*\l@figure{\@dottedtocline{1}{0em}{2.3em}}% Default: 1.5em/2.3em
\let\l@table\l@figure
\makeatother

%Abbildungsverzeichnis
\newcommand{\skpciteprintfig}{
	\DeclareCiteCommand{\tabcite}[\mkbibbrackets]
		{\usebibmacro{prenote}}
		{\usebibmacro{citeindex}%
		 \usebibmacro{cite}}
		{\multicitedelim}
		{\usebibmacro{postnote}}

	\renewcommand{\addvspace}[1]{}
	\let\oldnumberline\numberline
	\renewcommand{\numberline}{\tofnumberline}

	\listoffigures

	\label{abbildungsverzeichnis}
}


%Bildunterschrift (\skpcitecaption{kurzer Name für das Abb. Verzeichnis}{Bildunterschrift}{Name der Quelle im .bib file})
\newcommand{\skpcitecaption}[3]{
  \caption[\textit{#1}. \fullcite{#3}]{#2\tabcite{#3}}
}
